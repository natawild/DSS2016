\chapter{Conclusão}

Para concluir a primeira fase de modelação do projeto que consistiu em  desenvolver um sistema de suporte à partilha de despesas num apartamento, foi  proposto enquadrar e descrever da forma mais detalhada possível o sistema a ser desenvolvido. Procedemos a uma primeira fase de investigação sobre como funciona a partilha de despesas dentro de uma casa /apartamento, investigação essa que foi relativamente simples dado que alguns dos elementos do grupo estão familiarizados com o problema em questão. A partir daí fizemos uma descrição do processo de análise de requisitos construindo assim o modelo de domínio onde foi utilizada a linguagem UML. Do modelo de domínio e requisitos do sistema foi possível desenvolver os diagramas de 'Use Case' e a posterior especificação de cada um deles. 

Depois de todos estes elementos procedemos à fase de pensar de como seria a nossa aplicação fisicamente falando e então para tal utilizamos o programa 'Pencil' e desta forma desenvolvemos a nossa primeira proposta de interface com o utilizador. Construímos também os diagramas de máquinas de estado de acordo com a interface pensada.

Um dos principais problemas que encontramos foi a modelação do Modelo de Domínio, pois estavam sempre a surgir novos requisitos e a serem eliminados outros assim como nas especificações dos 'Use Case'. Podemos concluir que esta é uma etapa que requer uma análise muito cuidada, pois estão sempre a surgir novas maneiras de abordar o projeto. 


Para a fase final do projeto desenvolvemos os "Diagramas de Sequência" com o objetivo de descrever o funcionamento da aplicação e percebemos que alguns "use cases" precisavam de ser alterados, na parte das especificações pois a maneira pensada inicialmente não estava bem conseguida.  Depois de muita discussão e partilha de ideias concebemos a aplicação com o objetivo de ser eficiente e simples de usar, com uma interface apelativa, até porque se idealizarmos algo demasiado complexo, muito provavelmente a escassez de tempo juntamente com a possível falta de conhecimentos, poderiam levar a uma má implementação ou até mesmo à impossibilidade de concluirmos o trabalho.
A solução usada para o problema pareceu-nos a melhor e fácil de implementar, contudo, esta solução está dependente do fator seriedade dos moradores da casa, pois existem certas funcionalidades que podem causar certa controvérsia. Por exemplo: quando um morador empresta dinheiro à casa poderá nunca o vir a receber, visto que aplicação só gere despesas. 

Em relação a trabalho futuro poderá ser implementado um sistema que permita a troca de mensagens entre os utilizadores e não só o administrador a enviar mensagens, pois a comunicação deverá ser feita entre todos os intervenientes do grupo. Além disso, convém ao administrador saber seuma mensagem foi vista por todos, ou quantos/quem ainda não viu essa mensagem, para que assim o administrador possa posteriormente apagar a mensagem pois já não tem utilidade. 

Outro aspeto a melhorar é o caso de o transferir dinheiro de uma conta para outra, ficando assim uma dívida direta. Também na apresentação das contas/mensagens o utilizador visualiza a mensagem/conta e só quando clicar é que consegue ver o detalhe das mesmas. 

Aplicação foi submetida a testes, e  corrigida quando os testes falharam no entanto há sempre testes que não conseguimos até ao momento prever. 
