\chapter{Conclusão}

Para concluir esta primeira fase de modelação do projeto que consiste em  desenvolver um sistema de suporte à partilha de despesas num apartamento, foi-nos proposto enquadrar e descrever da forma mais detalhada possível o sistema a ser desenvolvido. Para isso fizemos uma descrição do processo de análise de requisitos construindo assim o modelo de domínio.  Do modelo de domínio e requisitos do sistema foi possível desenvolver os diagramas de 'Use Case' e a posterior especificação de cada um deles. 

Depois de todos estes elementos procedemos à fase de pensar de como seria a nossa aplicação fisicamente falando e então para tal utilizamos o programa 'Pencil' e desta forma desenvolvemos a nossa primeira proposta de interface com o utilizador. Construimos também os diagramas de máquinas de estado de acordo com a interface pensada. Porém estes dois pontos ainda têm muitas arestas para limar, pois ainda não temos uma interface completa com todas as possiveis funcionalidades. 

Um dos principais problemas que encontramos foi a modelação do Modelo de Dominio, pois estavam sempre a surgir novos requisitos e a serem eliminados outros assim como nas especificações dos 'Use Case'. Podemos concluir que esta é uma etapa que requer uma análise muito cuidada, pois estão sempre a surgir novas maneiras de abordar o projeto. 


Para a fase final do projeto desenvolvemos os "Diagramas de Sequência" com o objetivo de descrever o funcionamento da aplicação e percebemos que alguns "use cases" precisavam de ser alterados.  Depois de muita discussão e partilha de ideias concebemos a aplicação com o objetivo de ser eficiente e simples de usar, com uma interface apelativa. 
A solução usada para o problema pareceu-nos a melhor e fácil de implementar, contudo esta solução está dependente da seriedade dos moradores da casa, por exemplo quando um morador empresta dinheiro poderá nunca o vir a receber. 
Em relação a trabalho futuro fica implementar um sistema que permita a troca de mensagens entre os utilizadores e não só o administrador a enviar mensagens, o administrador saber que aquela mensagem foi vista por todos , ou quantos ainda não vieram essa mensagem, de maneira a que o admin saiba quem viu ou não a mensagem e consequentemente poder apagá-la. 
Outro aspeto a melhorar é o caso de o utilizador emprestar dinheiro ao utilizador diretamente da conta dele. 
Na apresentação das contas/mensagens o utilizador primeiro vê as mensagens e só quando carregar é que consegui ver o detalhe das mesmas. 
Aplicação foi submetida a testes, e  corrigida quando os testes falhavam no entanto há sempre testes que não conseguimos até ao momento prever. 




