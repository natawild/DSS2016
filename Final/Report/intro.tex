\chapter{Introdução}

No âmbito da unidade curricular de Desenvolvimento de Sistemas de Software do 3ºano do curso de MIEI,  foi proposto o desenvolvimento de um projeto que visa a concepção de um sistema que serve de suporte à partilha de despesas num apartamento.

Neste relatório é descrito o processo de análise, modelação e concepção de um sistema que serve de suporte à partilha de despesas num apartamento. Foi-nos proposto desenvolver uma aplicação que fosse capaz de fazer o registo das despesas que são geradas num apartamento, assim como a gestão do pagamento feita por cada um dos moradores do apartamento em questão.
Decidimos que seria interessante desenvolver um sistema que permitisse aos utilizadores usufruírem do controlo de terem as suas despesas devidamente divididas, onde os pagamentos das mesmas fossem o mais breve possível, organizado e ainda a facilidade de acesso através de um smartphone, tablet ou pc para a consulta desta mesma aplicação.

O trabalho será dividido em duas fases que se completam uma à outra.
Na primeira fase será descrito o processo de análise de requisitos suportada pelo modelo de domínio do sistema, casos de uso e respectivas especificações e uma possível interface com o utilizador. Tudo isto de forma a enquadrar e descrever da forma mais detalhada possível o sistema a ser desenvolvido. Serão expostos os desenhos de planificação das interfaces e da sua correlação com as funcionalidades a serem implementadas no sistema.

Nesta segunda fase faremos com que a nossa aplicação ganhe vida e desta forma conseguirmos fazer chegar ao público alvo aquilo que seria o nosso produto final. Assim sendo, será necessário:

\begin{itemize}
	\item criar uma base de dados que irá conter informação sobre os utilizadores da aplicação, bem como outras informações úteis como o histórico de despesas que deverá estar sempre atualizado;
	
	\item criar uma interface apelativa e funcional, a qual será criada à semelhança dos Mockups criados e apresentados na primeira fase; 
	
	\item fazer a ligação dessa mesma interface à Base de Dados criada
\end{itemize}

Para além disso serão desenhados os modelos de Dominio, os Modelos de Domínio com DAO, os diagramas de package, os diagramas de instalação e os diagramas de atividade. Todos estes modelos servirão de base para a implementação do projeto. 








\section{Apresentação do Caso de Estudo}

A aplicação terá como objetivo desenvolver um sistema de despesas num apartamento  capaz de suportar o registo de despesas e a gestão do pagamento dessas mesmas despesas por parte dos moradores. Este é um sistema que proporciona aos seus utilizadores a possibilidade de efetuarem os pagamentos das suas despesas, sejam estas recorrentes (por exemplo, água ou eletricidade) ou extraordinárias (por exemplo, necessidade de realizar alguma reparação no apartamento) fazendo com que o controlo de dívidas entre moradores estejam sempre atualizadas e visíveis para todos os utilizadores da aplicação. 	
Neste sistema existe um morador previamente registado na aplicação necessitando de fornecer o nome, data de nascimento, e-mail e número de telefone/telemóvel e uma password para efetuar o registo. Terá a ele associada uma conta corrente que será uma espécie de fundo do qual mensalmente (ou quando necessário) é creditado o pagamento, ou seja, a quantia necessária afeta as despesas correspondentes. A conta corrente de cada morador contribui para o saldo global, ou seja, no nosso sistema existe um saldo que resulta do somatório de todas as contas correntes e que corresponde ao montante total que o apartamento tem como despesa nesse mês. 

O saldo global é administrado por um Administrador que é responsável por verificar se o montante deste mesmo está completo e depois dessa forma, pagar a despesa. A todas as despesas está associado um valor assim como cada pagamento creditado da conta corrente de cada morador. 

Ainda de salientar que a cada morador está associada uma fração que varia consoante o tipo de morador (exemplo: moradores que partilham quarto, terão uma fração do valor da renda menor relativamente a um morador com quarto individual), assim como relativamente às restantes despesas (água, luz, gás, internet, ou até mesmo despesas extraordinárias). 


 	