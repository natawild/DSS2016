\chapter{Conclusão}

Para concluir esta primeira fase de modelação do projeto que consiste em  desenvolver um sistema de suporte à partilha de despesas num apartamento, foi-nos proposto enquadrar e descrever da forma mais detalhada possível o sistema a ser desenvolvido. Para isso fizemos uma descrição do processo de análise de requisitos construindo assim o modelo de domínio.  Do modelo de domínio e requisitos do sistema foi possível desenvolver os diagramas de 'Use Case' e a posterior especificação de cada um deles. 

Depois de termos todos estes elementos procedemos à fase de pensar de como seria a nossa aplicação fisicamente falando e então para tal utilizamos o programa 'Pencil' e desta forma desenvolvemos a nossa primeira proposta de interface com o utilizador. Construimos também os diagramas de máquinas de estado de acordo com a interface pensada. Porém estes dois pontos ainda têm muitas arestas para limar, pois ainda não temos uma interface completa com todas as possiveis funcionalidades. 

Um dos principais problemas que encontramos foi a modelação do Modelo de Dominio, pois estavam sempre a surgir novos requisitos e a serem eliminados outros assim como nas especificações dos 'Use Case'. 





